\section{Power Reactor Terminology}
\begin{labeling}
    \item [\underline{Coolant}:] Material used to remove heat from core, to
        heat water, to push a turbine, etc.
    \item [\underline{Steam or Coolant Loops}:] Number of heat transfer mechanisms.
        Must be at least 1.
    \item [\underline{Moderator}:] Material that slows down neutrons. Not all
        reactors are moderated. Sometimes this is the same as the coolant. Typically
        low-Z materials.
    \item [\underline{Neutron Energies}:]
        \begin{labeling}
            \item [\underline{Fast}:] At or near fission neutron birth energies $> 1$ MeV
            \item [\underline{Epithermal}:] Neutrons in the process of slowing down
                $[10^{-5}, 1)$ MeV.
            \item [\underline{Thermal}:] Neutrons in thermal equilibrium with the moderator,
                $< 10^{-5}$ MeV. Maxwellian distribution
                % FIXME Add image of maxwell distribution for linaer and log.
        \end{labeling}
    \item [\underline{Fuel Production}:]
        \begin{labeling}
            \item [\underline{Burner}:] If destroies more atomes of fuel at discharge.
            \item [\underline{Breeder}:] If creates more atoms of fuel at discharge
        \end{labeling}
\end{labeling}

% FIXME Add single loop core image
% FIXME Add two loop core image

\section{Neutron Multiplication}
Reactors are controlled by how many neutron are in teh core at any given time.
This is because neutrons cause fission, which releases energy and more neutrons.

\section{Prompt Neutron Multiplication}
``Prompt'' means that neutrons born in a given generation are destroyed or lost before the
next generation, i.e. instantly.

Change in \# neutrons = Sources - Losses

Symbolically,
\[ \frac{dN}{dt} = \nu \Sigma_f\phi - \Sigma_a\phi -DB^2\phi \]

Let's go through this term-by-term.

Note: The book uses $\Phi$ for $\phi$

\section{Difussion Theory Terms}
\begin{labeling}
    \item [$N$:] Neutron density [n/cm$^3$]
    \item [$t$:] Time [sec]
    \item [$\nu$:] Number of neutrons born per fission, 2 - 5 typically [unitless]
    \item [$\Sigma_f$:] macroscopic fission cross section, [1/cm]
    \item [$\phi$:] neutron flux, [n/cm$^2$ s]
    \item [$\Sigma_a$:] Macroscopic absorption cross-section
    \item [$D$:] Diffission coefficient, [cm], related to neutron mean free path
        $\lambda_{\mathrm{tr}}$ by
        \[ D = \frac{\lambda_{\mathrm{tr}}}{3} = \frac{1}{3\Sigma_{\mathrm{tr}}} = \frac{1}{3\Sigma_s(1 -\frac{2}{3A})} \]
    \item [$B$:] Buckling [1/cm]
        \begin{itemize}
            \item \mathbf{Geometric Buckling} $B_g$ depends on the shape of the core, see table 4-2.
            \item \mathbf{Material Buckling} $B_m$'s is derived from the criticality constraint
                such that $B_m^2 = \frac{\nu\Sigma_f - \Sigma_a}{D}$ for critical systems.
        \end{itemize}
\end{labeling}

So, $\frac{dN}{dt}$ is the growth rate.

$\nu\Sigma_f\phi$ is the rate of neutron birth.

$\Sigma_a\phi$ is the rate at which neutrons are destroyed.

$DB^2\phi$ is the \textit{Leakage}, or the rate at which neutrons leave the system entirely.

\underline{On the board:}  Units for each term. Ensure that they are the same.
