\section{Conceptual Fission}
Comercial nuclear power is fission-based.

The word ``fission'' comes from tge Latin word for ``to split''.

% FIXME Image for Spontaneous fission

% FIXME Image for Neutron-induced fission

% FIXME Image for Delayed fission

% FIXME Image for Transmutation, then fission

Fussion, on the other hand, means ``to stick''. In some ways it is the reverse of
fission.

% FIXME Image for Fussion

Incident neutrons have only some probability $p$ or $p(\vec{v}, N_i)$ of interacting with
the material. We won't go into this too much right now.

\underline{Question for the board:} Draw a chain reaction.

Nuclides can be classified in the following ways.
\begin{labeling}
\item [\underline{fissile}:] Can sustain a chain reaction with only this nuclide, e.g. \nuc{U}{235}.
\item [\underline{fissionable} or \underline{fertile}:] Can be split, but needs initial neutrons to get
    chain reaction started and to keep going indefinitely, e.g. \nuc{U}{238}.
\item [\underline{non-fissionable} or \underline{inert}:] Won't fission, e.g. \nuc{H}{1}.
\end{labeling}

\section{Intro to the Nuclear Fuel Cycle}
Nuclear power is ``hot rock in water'' technology. The catch here is that the choice of rock and the
choice of water matters.

The NFC is made up of different facilities that create fresh fuel, generate power, and manage used fuel.

The most common fuel cycle is a uranium-based ``Once-Through'' FC. This is because fuel atoms only go through
a reactor once.

% FIXME Image for Once Through FC

% FIXME Image for Pu/U Recycle

What are the effects of recyling on,
\begin{itemize}
\item NU mined?
\item Cost?
\item Repository space?
\end{itemize}

